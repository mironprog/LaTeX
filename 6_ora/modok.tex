\documentclass{article}
\usepackage[magyar]{babel}
\usepackage{t1enc}
\usepackage{amsmath}
\usepackage{amsfonts, amssymb}
\usepackage{mathtools}
\usepackage{graphicx}

\begin{document}
{\huge 1. Bevezető}\\

Az $\frac{1}{n^2}$ sorösszege:
\[ \sum_{i=1}^\infty \frac{1}{n^2} = \frac{\pi^2}{6}. \]\\
Az n! (n faktoriális) a számok szorzata 1-töl n-ig, azaz
\[ n! := \prod_{k=1}^n k = 1 \cdot 2 \cdot \dotsc \cdot n.\]
Konvenció szerint 0! = 1\\


Legyen 0   k   n. A binomiális együttható
\[\binom{n}{k} := \frac{n!}{k!\cdot(n-k)!} \]
ahol a faktoriálist (1) szerint definiáljuk.\\

Az előjel- azaz szignum függvényt a következőképpen definiáljuk:
\[sgn(x) :=
\begin{cases}
1, & \text{ha } x > 0, \\
0, & \text{ha } x = 0, \\
-1, & \text{ha } x < 0.
\end{cases}
\]

{\huge 2.Determináns}\\

Legyen
\[[n] := \{1,2, \dotsc , n \} \]
a természetes számok halmaza 1-től n-ig.\\

Egy n-edrendű permutáció $\sigma$ egy bijekció [n]-ből [n]-be. Az n-edrendű permutációk halmazát, az ún. szimmetrikus csoportot, $S_n$-nel jelöljük.\\

Egy $\sigma \in  S_n$ permutációban inverziónak nevezünk egy (i, j) párt, ha i < j
de $\sigma_i > \sigma_j$.\\

Egy $\sigma \in  S_n$ permutáció paritásának az inverziók számát nevezzük:
\[ \mathcal{I} (\sigma) := \vert \{(i, j) \vert i, j \in [n], i < j, \sigma_i > \sigma_j \} \vert \]\\

Legyen A $\in$ $R^{n \times n}$, egy n \texttimes n-es (négyzetes) valós mátrix:\\

\[ A=
\left( \begin{matrix}
a_{11} & a_{12} & \dotsc  & a_{1n} \\
a_{21} & a_{22} &  \dotsc & a_{2n} \\
\rotatebox[origin=c]{90}{$ \dotsc $} & \rotatebox[origin=c]{90}{$ \dotsc $} &  \rotatebox[origin=c]{-45}{$ \dotsc $} & \rotatebox[origin=c]{90}{$ \dotsc $ } \\
a_{n1} & a_{n2} & \dotsc & a_{nn} \\
\end{matrix} \right)
\]\\

Az A mátrix determinánsát a következőképpen definiáljuk:
\[ det(A) = 
\begin{vmatrix}
a_{11} & a_{12} & \dotsc  & a_{1n} \\
a_{21} & a_{22} &  \dotsc & a_{2n} \\
\rotatebox[origin=c]{90}{$ \dotsc $} & \rotatebox[origin=c]{90}{$ \dotsc $} &  \rotatebox[origin=c]{-45}{$ \dotsc $} & \rotatebox[origin=c]{90}{$ \dotsc $ } \\
a_{n1} & a_{n2} & \dotsc & a_{nn} \\
\end{vmatrix}
:=  \sum_{\sigma  \in\ S_n} (-1)^{ \mathcal{I} (\sigma)} \prod_{i=1}^n a_{i\sigma_i} \]\\

{\huge 3. Logikai azonosság}\\


Készítsük el a következő táblázatot, formulákat és levezetést. Ügyeljünk a számozásra és igazításra is!
Tekintsük az L = \{0, 1\} halmazt, és rajta a következő, igazságtáblával definiált
műveleteket:\\
\[
\begin{array}{c||c}
x & \bar{x}   \\ \hline
0 & 1  \\
1 & 0  \\
\end{array}
\quad \quad \quad
\begin{array}{c c||c|c|c}
x & y & x \vee y & x \wedge y & x \rightarrow y \\ \hline
0 & 0 & 0 & 0 & 1 \\
0 & 1 & 1 & 0 & 1 \\
1 & 0 & 1 & 0 & 0 \\
1 & 1 & 1 & 1 & 1 \\

\end{array}
\]


Legyenek a, b, c, d $\in$ L. Belátjuk a következő azonosságot:

\begin{gather}
(a \wedge b \wedge c) \rightarrow d = a \rightarrow (b \rightarrow (c \rightarrow d)).
\end{gather}

\[
(a+b)^{n+1}
= (a+b) \cdot \left( \sum_{k=0}^n \binom{n}{k} a^{n-k}b^k \right)
= \sum_{k=0}^n \binom{n}{k} a^{(n+1)-k}b^k
+ \sum_{k=1}^{n+1} \binom{n}{k-1} a^{(n+1)-k}b^{k}
= \binom{n+1}{0} a^{n+1-0} b^0
+ \sum_{k=1}^n \binom{n+1}{k} a^{(n+1)-k}b^k
+ \binom{n+1}{n+1} a^{n+1-(n+1)} b^{n+1}
= \sum_{k=0}^{n+1} \binom{n+1}{k} a^{(n+1)-k}b^k
\]
\end{document}