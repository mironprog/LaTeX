\documentclass[aspectratio=169, 12pt, xcolor={table}]{beamer}
\usepackage[magyar]{babel}
\usepackage{t1enc}
\usepackage{hyperref}
\usepackage{xcolor}
\usepackage{amsthm}
\usepackage{hulipsum}
\usepackage{graphicx}


\author{Miron} 
\title{Dia} 
\subtitle{Diaalcím} 
\institute{Miskolci egyetem} 
\date{\today} 

\begin{document}


ihjerwphiwj

\begin{frame}
\maketitle
\end{frame}

\begin{frame}
\frametitle{Title}
\framesubtitle{Subtitle}
\hulipsum[1]
\end{frame}

\begin{frame}{Probaframe}{Probaalcim}
\begin{itemize}
\item framenumber: \insertframenumber \\ 
slidenumber: \insertslidenumber
\end{itemize}
\end{frame}

\begin{frame}{Frame, slide példa}
\begin{itemize}
\item itt vagyunk most: \insertframenumber.\ frame,
\insertslidenumber.\ slide
\item ez az elem minden slide-on megjelenik
\item<1> ez az elem csak az els® slide-on
jelenik meg
\item<2-> ez az elem csak a második slide-tól
jelenik meg
\item \alert<3>{ez az elem piros a 3.\ slide-on}
\item összesen tehát 3 slide lesz
\end{itemize}
\end{frame}

\begin{frame}[fragile]
\begin{verbatim}
\hulipsum
\end{verbatim}
\end{frame}

\begin{frame}[allowframebreaks]
\hulipsum
\end{frame}

\begin{frame}
\begin{columns}[c]
\begin{column}{.5\linewidth}
\begin{itemize}
  \item első elem
  \item második elem
  \item harmadik elem
\end{itemize}
\begin{enumerate}
  \item első számozott
  \item második számozott
  \item haramdik számozott
\end{enumerate}
\end{column}
\begin{column}{.5\linewidth}
\begin{figure}[bt]
\caption{Egy süni}
\label{fig:kepek}
\includegraphics[width=4cm, height=4cm]{C:/Users/Diak/Desktop/8_ora/forras/suni.jpg}
\end{figure}
\end{column}
\end{columns}
\end{frame}

\begin{frame}

\begin{block}{Blokk címe}
Ez itt egy blokk
\begin{itemize}
\item egy elem
\end{itemize}
\end{block}

\begin{block}{}
Ez itt egy cím nélküli blokk
\end{block}

\begin{exampleblock}{Exampleblokk címe}
Ez itt egy exampleblokk
\begin{itemize}
\item egy elem
\end{itemize}
\end{exampleblock}

\begin{alertblock}{Alertblokk címe}
Ez itt egy alertblokk
\begin{itemize}
\item egy elem
\end{itemize}

\end{alertblock}


\end{frame}

\begin{frame}
\begin{theorem}
Valami
\end{theorem}

\begin{proof}
Megintcsak valami
\end{proof}

\begin{semiverbatim}
\\begin \{enumerate\}  \newline
\\item \alert{első szint}  \newline
\\begin \{enumerate\} \newline
\\item \textcolor{blue}{második szint} \newline
\\end \{enumerate\} \newline
\\end \{enumerate\}
\end{semiverbatim}
\end{frame}




\end{document}