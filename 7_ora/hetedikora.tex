\documentclass{article}
\usepackage[magyar]{babel}
\usepackage{t1enc}
\usepackage{amsmath}
\usepackage{amsfonts, amssymb}
\usepackage{mathtools}
\usepackage{graphicx}
\usepackage{ifthen}
\usepackage{pgffor}
\usepackage{multicol}
\usepackage{hulipsum}




\DeclareMathOperator *{\argmax}{arg\max}

\DeclarePairedDelimiter{\ceil}{\lceil}{\rceil}

\DeclarePairedDelimiterX{\szogfelt}[2]{[}{]}{#1\;\delimsize\vert\;#2}
\newcommand{\probcond}[2]{\prob\szogfelt{#1}{#2}}

\DeclarePairedDelimiter{\szogletes}{[}{]}
\newcommand{\VE}[1]{\mathbb{E}\szogletes*{#1}}

\newenvironment{vonalzott}[1]{\begin{minipage}{0.8\linewidth}
\vspace{1ex}\hrule\vspace{1ex}{\huge\begin{center}#1\end{center}}}%
{\vspace{1ex}\hrule\vspace{1ex}\end{minipage}}

\newcounter{szamlalo}


\begin{document}
$\argmax_{x\in[0,1]}$
\\
\\
\\
x*:=$\argmax_{x\in[0,1]}$ x $log_2$ (x).
\\
\\
\\
$\ceil{x}, \ceil{\dfrac{5}{3}}$
\\
\\
\\
$\szogfelt{\sum_{i=1}^N X_i = x}{N = n}$
\\
\\
\\
$\VE{\sum\limits_{i=1}^N X_i}$ = 

\vspace{4cm}



\begin{vonalzott}{Kulcsgondolatok}



\hulipsum[1]
\\
\newcommand{\kgitem}{\par\stepcounter{szamlalo}\makebox[10pt][c]{\theszamlalo}}
\kgitem

\end{vonalzott}






\end{document}