\documentclass[twoside,12pt]{article}
\usepackage[magyar]{babel}
\usepackage{t1enc}
\usepackage{xcolor}
\usepackage{blindtext}
\usepackage{hulipsum}
\usepackage{lipsum}
\usepackage{geometry}
\usepackage{fancyhdr}

\geometry{inner=3cm,outer=5cm}
\geometry{bindingoffset=1cm}
\geometry{marginparwidth=3cm,marginparsep=0.5cm}
\geometry{headheight=15pt}

\setcounter{tocdepth}{7}
\renewcommand{\thefootnote}{\fnsymbol{footnote}}


\title{Második óra}
\author{Miron}
\date{\today}

\begin{document}
\pagestyle{fancy}
\fancyhead[LE,RO]{\thepage}
\fancyfoot[C]{Miskolci egyetem}
\renewcommand{\footrulewidth}{0.4pt}




\maketitle

\begin{abstract}
\hulipsum[1]

\hulipsum[2]
\footnote{}
\end{abstract}

\pagenumbering{roman}
\tableofcontents
\pagenumbering{arabic}
\pagebreak




\section{első szakasz}
\footnote{}
\subsection{}
\hulipsum[1]



\subsection{}
\hulipsum[1]

\vspace{8cm}
\section[masodik szakasz]{második szakasz hosszú neve}

\paragraph{1.3}
\hulipsum[1]
\subparagraph{1.4}
\hulipsum[1]


\appendix
\section{harmadik szakasz}
\subsection{}
\subsection{}

\section{negyedik szakasz}
\subsection{}
\subsection{}

\begin{quote}
\lipsum[1]
\end{quote}

\begin{quotation}
\lipsum[1]
\lipsum[2]
\end{quotation}

\begin{verse}
Ne félj már úgy, legyél bátrabb!
Ha kicsit fáj, nehogy rángasd!
Kapsz gyógypuszit, hozd a nyuszit!
Hé, Anya, Te nem kapsz szurit!

Doktor bácsi, ugye, látja,
milyen bátor a kis Sára?
Nem Ő reszket, csak a nyuszi.
Vigasztald meg... Jaj! A szuri!

Forrás: www.poet.hu magyar versek
\end{verse}

\end{document}