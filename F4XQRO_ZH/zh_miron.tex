\documentclass[11pt]{book}
\usepackage[magyar]{babel}
\usepackage{t1enc}
\usepackage{hulipsum}
\usepackage{graphicx}
\usepackage{fancyhdr}
\usepackage{subcaption}
\usepackage{wrapfig}
\usepackage{amsmath}
\usepackage{amsfonts, amssymb}
\usepackage{mathtools}
\usepackage{dsfont}
\usepackage{xcolor}
\usepackage{listings}

\newenvironment{vers}[2]{\begin{center}\huge{#1}\end{center},\begin{center}\emph{#2}\end{center}}


\title{Zárthelyi dolgozat}
\author{Miron F4XQRO}
\date{\today}


\begin{document}
\maketitle
\pagestyle{fancy}
\fancyhead[LE]{\leftmark}
\fancyhead[RO]{\rightmark}
\fancyhf[]{\nouppercase{\leftmark}}
\fancyfoot[LE,RO]{\thepage}
\chapter{első feladat}
\section{első szakasz}
\hulipsum
\section{második szakasz}
\hulipsum


\chapter{második feladat}
\includegraphics[keepaspectratio,scale=0.5,height=4cm]{C:/Users/Diak/Desktop/zh/kepek/szines.jpg}
\framebox{\includegraphics[height=4cm]{C:/Users/Diak/Desktop/zh/kepek/szepia.jpg}}


\begin{figure}[hb]
\centering
\caption{ábra felirata}
\begin{subfigure}{5cm}
\centering
\caption{első részábra}
\includegraphics[keepaspectratio,scale=0.5,height=4cm]{C:/Users/Diak/Desktop/zh/kepek/szines.jpg}
\end{subfigure}

\begin{subfigure}{5cm}
\centering
\caption{második részábra}
\framebox{\includegraphics[height=4cm]{C:/Users/Diak/Desktop/zh/kepek/szepia.jpg}}
\end{subfigure}
\end{figure}
\hulipsum

\begin{wrapfigure}[12]{r}[\marginparwidth]{0pt}
\hulipsum
\includegraphics[keepaspectratio,scale=0.5,height=4cm]{C:/Users/Diak/Desktop/zh/kepek/szines.jpg}
\framebox{\includegraphics[height=4cm]{C:/Users/Diak/Desktop/zh/kepek/szepia.jpg}}
\hulipsum

\end{wrapfigure}


\chapter{harmadik feladat}

{\huge 1.Definíció} (Sajátérték). Legyen A $\in \mathds{R}^{nxn}$ 
négyzetes mátrix. Azt mondjuk,
hogy $\lambda\in\mathds{c}$ C sajátértéke és v$\in\mathds{c}^n$ a $\lambda$ sajátértékhez tartozó (jobb oldali) sajátvektora A-nak, ha \\
\begin{center}Av = $\lambda$v.\end{center}

{\huge 2.Definíció} (Karakterisztikus polinom). Jelölje E$\in \mathds{R}^{nxn}$egységmátrixot.
Az A ún. \emph{karakterisztikus polinomja}:
\[ \varphi ( \lambda ):= det(A \textcolor{red}{- \lambda E}) = 
\begin{vmatrix}
a_{11} \textcolor{red}{- \lambda} & a_{12} & \dotsc  & a_{1n} \\
a_{21} & a_{22} \textcolor{red}{- \lambda} &  \dotsc & a_{2n} \\
\rotatebox[origin=c]{90}{$ \dotsc $} & \rotatebox[origin=c]{90}{$ \dotsc $} &  \rotatebox[origin=c]{-45}{$ \dotsc $} & \rotatebox[origin=c]{90}{$ \dotsc $ } \\
a_{n1} & a_{n2} & \dotsc & a_{nn} \textcolor{red}{- \lambda} \\
\end{vmatrix}
\]
egy n-edfokú polinom $\lambda$-ban.\\
\\

{\huge 1.Tétel}(Sajátértékek meghatározása). Az A $\in \mathds{R}^{nxn}$ mátrix sajátértékei az ún.
\emph{karakterisztikus egyenlet}
\begin{center} $\varphi(\lambda)$ = 0\end{center}
megoldásai. Mivel a $\varphi(\lambda)$ karakterisztikus polinom egy n-edfokú polinom $\lambda$-ban,
ezért a komplex számokon (multiplicitással együtt) \emph{n} megoldása van

\chapter{negyedik feladat}
\hulipsum
\begin{lstlisting}[language=C, tabsize=2, numbers=right, stepnumber=2, showspaces,
frame=shadowbox, float, caption={bináris keresés C-ben}]{C:/Users/Diak/Desktop/zh/kepek/binsearch_it.c}
binarySearch(arr, x, low, high)
	repeat till low = high
		mid = (low + high)/2
			if (x == arr[mid])
				return mid

			else if (x > arr[mid])	// x is on the right side
				low = mid + 1

			else			// x is on the left side
				high = mid - 1
\end{lstlisting}
\hulipsum

\chapter{ötödik feladat}
\begin{vers}{Cím}{Szerzo}

\end{vers}

\end{document}